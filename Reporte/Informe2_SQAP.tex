\documentclass[12pt,a4paper]{article}

% Paquetes necesarios
\usepackage[utf8]{inputenc}
\usepackage[spanish]{babel}
\usepackage{graphicx}
\usepackage{geometry}
\usepackage{amssymb}
\usepackage{fancyhdr}
\usepackage{titlesec}
\usepackage{enumitem}
\usepackage{hyperref}
\usepackage{xcolor}
\usepackage{listings}
\usepackage{newunicodechar}
\newunicodechar{⚠}{\textwarning} % or replace with a suitable LaTeX symbol
\usepackage{tabularx}
\usepackage{multirow}
\usepackage{booktabs}
\usepackage{float}
\usepackage{longtable}
\usepackage{pdflscape}
\usepackage[backend=biber,style=ieee,sorting=none]{biblatex}

% Configuración de márgenes
\geometry{top=2.5cm, bottom=2.5cm, left=3cm, right=2.5cm}
\setlength{\headheight}{14pt}

% Configuración de encabezados y pies de página
\pagestyle{fancy}
\fancyhf{}
\fancyhead[L]{\small Diseño de Pruebas - TravelBrain}
\fancyhead[R]{\small ESPE 2026}
\fancyfoot[C]{\thepage}
\renewcommand{\headrulewidth}{0.4pt}
\renewcommand{\footrulewidth}{0.4pt}

% Configuración de hipervínculos
\hypersetup{
    colorlinks=true,
    linkcolor=blue,
    filecolor=magenta,      
    urlcolor=cyan,
    citecolor=blue,
    pdftitle={Diseño de Pruebas - TravelBrain},
    pdfauthor={Cáceres Germán, Anthony Villareal},
}

% Configuración de colores para código
\definecolor{codegreen}{rgb}{0,0.6,0}
\definecolor{codegray}{rgb}{0.5,0.5,0.5}
\definecolor{codepurple}{rgb}{0.58,0,0.82}
\definecolor{backcolour}{rgb}{0.95,0.95,0.92}

% Definir lenguaje JavaScript para listings
\lstdefinelanguage{JavaScript}{
keywords={typeof, new, true, false, catch, function, return, null, catch, switch, var, if, in, while, do, else, case, break, const, let, async, await, class, extends, export, import, require},
keywordstyle=\color{blue}\bfseries,
ndkeywords={class, export, boolean, throw, implements, import, this},
ndkeywordstyle=\color{darkgray}\bfseries,
identifierstyle=\color{black},
sensitive=false,
comment=[l]{//},
morecomment=[s]{/*}{*/},
commentstyle=\color{codegreen}\ttfamily,
stringstyle=\color{codepurple}\ttfamily,
morestring=[b]",
morestring=[b]'
}

\lstdefinestyle{mystyle}{
    backgroundcolor=\color{backcolour},   
    commentstyle=\color{codegreen},
    keywordstyle=\color{magenta},
    numberstyle=\tiny\color{codegray},
    stringstyle=\color{codepurple},
    basicstyle=\ttfamily\footnotesize,
    breakatwhitespace=false,         
    breaklines=true,                 
    captionpos=b,                    
    keepspaces=true,                 
    numbers=left,                    
    numbersep=5pt,                  
    showspaces=false,                
    showstringspaces=false,
    showtabs=false,                  
    tabsize=2,
    literate=
        {á}{{\'a}}1 {é}{{\'e}}1 {í}{{\'i}}1 {ó}{{\'o}}1 {ú}{{\'u}}1
        {Á}{{\'A}}1 {É}{{\'E}}1 {Í}{{\'I}}1 {Ó}{{\'O}}1 {Ú}{{\'U}}1
        {ñ}{{\~n}}1 {Ñ}{{\~N}}1
}

\lstset{style=mystyle}

% Archivo de bibliografía
\addbibresource{referencias.bib}

% Símbolos para estados (usando LaTeX seguro)
\newcommand{\estadook}{\textcolor{green}{$\checkmark$}}
\newcommand{\estadowarning}{\textcolor{orange}{\textbf{!}}}
\newcommand{\estadopending}{\textcolor{blue}{\textbf{...}}}
\newcommand{\estadofailed}{\textcolor{red}{\textbf{X}}}

\begin{document}

% ============================================================
% PORTADA
% ============================================================
\begin{titlepage}
    \centering
    \vspace*{1cm}
    
    {\Large \textbf{UNIVERSIDAD DE LAS FUERZAS ARMADAS ESPE}} \\[0.5cm]
    {\large Departamento de Ciencias de la Computación} \\[1.5cm]
    
    \includegraphics[width=0.3\textwidth]{img/espe.png} \\[1cm]
    
    {\LARGE \textbf{Proyecto Final: Plan de Aseguramiento de la Calidad (SQAP)}} \\[0.5cm]
    {\Large \textbf{Sistema TravelBrain}} \\[1.cm]
    
    {\large \textbf{Informe 2: Diseño de Casos de Prueba y Matrices de Rastreabilidad}} \\[1cm]
    
    {\large \textbf{ACTUALIZADO CON RESULTADOS DE EJECUCIÓN}} \\[0.5cm]
    {\small (Estado al 21 de enero de 2026)} \\[1.5cm]
    
    \begin{tabular}{ll}
        \textbf{Asignatura:} & Desarrollo De software Seguro \\
        \textbf{NRC:} & 27886 \\
        \textbf{Estudiantes:} & Cáceres Germán (Scrum Master \& Tech Lead) \\
                            & Ponce Diego (Development Team) \\
        \textbf{Docente:} & Ing. Angel Cudco, Mgs. \\
        \textbf{Fecha:} & 12 de febrero de 2026 \\
    \end{tabular}
    
    \vfill
    
    {\large Sangolquí, Ecuador} \\
    {\large 2026}
\end{titlepage}

% ============================================================
% ÍNDICE
% ============================================================
\tableofcontents
\newpage

% ============================================================
% LISTA DE TABLAS
% ============================================================
\listoftables
\newpage

% ============================================================
% RESUMEN EJECUTIVO
% ============================================================
\section*{Resumen Ejecutivo}
\addcontentsline{toc}{section}{Resumen Ejecutivo}

El presente documento constituye el \textbf{Diseño Detallado de Casos de Prueba} para el sistema TravelBrain, complementando el Plan Maestro de Pruebas establecido en el Informe 1.

Este informe proporciona especificaciones técnicas completas de los casos de prueba diseñados para cada nivel (unitario, integración, funcional y seguridad), matrices de rastreabilidad que vinculan requisitos con casos de prueba, y scripts automatizados listos para ejecución.

\subsection*{Estado Actual de Ejecución}
\textbf{Actualización basada en resultados reales obtenidos:}

\begin{itemize}
    \item \textbf{Pruebas E2E (Cypress):} 9 casos ejecutados exitosamente
    \item \textbf{Pruebas Unitarias Frontend (Jest + RTL):} 14 componentes y 7 servicios probados
    \item \estadowarning \textbf{Pruebas Unitarias Backend (Jest):} 10 conjuntos ejecutados con fallos
    \item \estadowarning \textbf{Pruebas Business Rules:} 5 conjuntos ejecutados con fallos
    \item \estadopending \textbf{Pruebas de Seguridad (OWASP ZAP):} Pendientes de ejecución
\end{itemize}

\subsection*{Métricas Clave Actuales:}
\begin{itemize}
    \item \textbf{Cobertura Total:} 15.3\% (SonarQube)
    \item \textbf{Issues Abiertos:} 381 (1 Seguridad, 94 Confiabilidad, 286 Mantenibilidad)
    \item \textbf{Tasa de Éxito Pruebas:} 34.5\% de 87 casos diseñados
    \item \textbf{Pruebas E2E Completadas:} 9/16 casos diseñados
\end{itemize}

\subsection*{Total de Casos Diseñados:} 
87 casos de prueba distribuidos en 4 niveles.

\subsection*{Próximos Pasos Críticos:}
\begin{enumerate}
    \item Corregir pruebas unitarias fallidas en backend
    \item Mejorar cobertura de código al menos al 70\%
    \item Ejecutar pruebas de integración y seguridad
    \item Resolver issues de confiabilidad (94 abiertos)
\end{enumerate}

\textbf{Palabras clave:} Casos de Prueba, Matriz de Rastreabilidad, Test Design, Cypress, Jest, Postman, OWASP ZAP, Ejecución Real, SonarQube.

\newpage

% ============================================================
% 1. INTRODUCCIÓN
% ============================================================
\section{Introducción}

\subsection{Propósito del Documento}

Este documento detalla el diseño completo de casos de prueba para el sistema TravelBrain, proporcionando especificaciones técnicas que servirán como guía durante la fase de ejecución del Sprint de Calidad.

\subsection{Alcance del Diseño}

El diseño cubre los siguientes niveles de prueba:

\begin{enumerate}
    \item \textbf{Pruebas Unitarias:} Componentes individuales (Frontend + Backend)
    \item \textbf{Pruebas de Integración:} APIs RESTful y comunicación entre servicios
    \item \textbf{Pruebas Funcionales E2E:} Flujos completos de usuario
    \item \textbf{Pruebas de Seguridad:} Vulnerabilidades OWASP Top 10
\end{enumerate}

\subsection{Referencias}

Este documento se basa en:

\begin{itemize}
    \item \textbf{Informe 1:} Plan Maestro de Pruebas (SQAP)
    \item \textbf{ARCHITECTURE.md:} Documentación técnica del sistema
    \item \textbf{IEEE 829:} Estándar para documentación de pruebas \cite{ieee829}
    \item \textbf{ISO/IEC/IEEE 29119:} Estándar de pruebas de software \cite{iso29119}
    \item \textbf{Resultados reales de ejecución:} Evidencias de Cypress, Jest y SonarQube
\end{itemize}

\subsection{Estado Actual de Ejecución}

Basado en las evidencias obtenidas, se presenta el siguiente estado de ejecución:

\begin{table}[H]
\centering
\caption{Resumen del Estado de Ejecución}
\label{tab:estado_resumen}
\begin{tabular}{|l|c|c|p{4cm}|}
\hline
\textbf{Tipo de Prueba} & \textbf{Diseñados} & \textbf{Ejecutados} & \textbf{Estado} \\ \hline
Unitarias (Frontend) & 15 & 21+ &  Completado \\ \hline
Unitarias (Backend) & 18 & 10+ & \estadowarning Con fallos \\ \hline
Pruebas E2E & 16 & 9 &  Completado \\ \hline
Integración & 28 & 0 & \estadopending Pendiente \\ \hline
Seguridad & 10 & 0 & \estadopending Pendiente \\ \hline
\textbf{TOTAL} & \textbf{87} & \textbf{40+} & \textbf{34.5\% completado} \\ \hline
\end{tabular}
\end{table}

% ============================================================
% 2. REQUISITOS FUNCIONALES
% ============================================================
\section{Requisitos Funcionales del Sistema}

\subsection{Módulo de Autenticación}

\begin{table}[H]
\centering
\caption{Requisitos Funcionales - Autenticación}
\label{tab:req_autenticacion}
\small
\begin{tabularx}{\textwidth}{|l|X|c|}
\hline
\textbf{ID} & \textbf{Requisito} & \textbf{Prioridad} \\ \hline
RF-AUTH-01 & El sistema debe permitir registro de usuarios con email, username y password & Alta \\ \hline
RF-AUTH-02 & El sistema debe validar formato de email (RFC 5322) & Alta \\ \hline
RF-AUTH-03 & El sistema debe requerir contraseñas con mínimo 8 caracteres & Alta \\ \hline
RF-AUTH-04 & El sistema debe hashear contraseñas con bcrypt antes de almacenar & Alta \\ \hline
RF-AUTH-05 & El sistema debe permitir login con email y password & Alta \\ \hline
RF-AUTH-06 & El sistema debe generar token JWT con expiración de 24h tras login exitoso & Alta \\ \hline
RF-AUTH-07 & El sistema debe rechazar credenciales inválidas con mensaje apropiado & Alta \\ \hline
RF-AUTH-08 & El sistema debe permitir logout y invalidar token JWT & Media \\ \hline
\end{tabularx}
\end{table}

\subsection{Módulo de Autenticación Biométrica}

\begin{table}[H]
\centering
\caption{Requisitos Funcionales - Biometría}
\label{tab:req_biometria}
\small
\begin{tabularx}{\textwidth}{|l|X|c|}
\hline
\textbf{ID} & \textbf{Requisito} & \textbf{Prioridad} \\ \hline
RF-BIO-01 & El sistema debe permitir registro de datos biométricos faciales & Alta \\ \hline
RF-BIO-02 & El sistema debe generar challenge token temporal (TTL 2 min) para verificación & Alta \\ \hline
RF-BIO-03 & El sistema debe extraer encoding 128D mediante microservicio facial & Alta \\ \hline
RF-BIO-04 & El sistema debe cifrar encodings con AES-256-CBC antes de almacenar & Alta \\ \hline
RF-BIO-05 & El sistema debe validar liveness score >= 0.6 & Alta \\ \hline
RF-BIO-06 & El sistema debe validar quality score >= 0.6 & Alta \\ \hline
RF-BIO-07 & El sistema debe comparar encodings con threshold 0.6 para autenticación & Alta \\ \hline
RF-BIO-08 & El sistema debe registrar auditoría de intentos biométricos & Media \\ \hline
\end{tabularx}
\end{table}

\subsection{Módulo de Gestión de Viajes}

\begin{table}[H]
\centering
\caption{Requisitos Funcionales - Viajes}
\label{tab:req_viajes}
\small
\begin{tabularx}{\textwidth}{|l|X|c|}
\hline
\textbf{ID} & \textbf{Requisito} & \textbf{Prioridad} \\ \hline
RF-TRIP-01 & El sistema debe permitir crear viajes con título, destino, fechas y presupuesto & Alta \\ \hline
RF-TRIP-02 & El sistema debe validar que startDate < endDate & Alta \\ \hline
RF-TRIP-03 & El sistema debe permitir listar viajes del usuario autenticado & Alta \\ \hline
RF-TRIP-04 & El sistema debe permitir actualizar datos de un viaje propio & Media \\ \hline
RF-TRIP-05 & El sistema debe permitir eliminar un viaje propio & Media \\ \hline
RF-TRIP-06 & El sistema debe impedir acceso a viajes de otros usuarios (IDOR protection) & Alta \\ \hline
\end{tabularx}
\end{table}

\subsection{Módulo de Clima}

\begin{table}[H]
\centering
\caption{Requisitos Funcionales - Clima}
\label{tab:req_clima}
\small
\begin{tabularx}{\textwidth}{|l|X|c|}
\hline
\textbf{ID} & \textbf{Requisito} & \textbf{Prioridad} \\ \hline
RF-WEAT-01 & El sistema debe buscar clima actual mediante OpenWeather API & Media \\ \hline
RF-WEAT-02 & El sistema debe almacenar historial de búsquedas de clima & Media \\ \hline
RF-WEAT-03 & El sistema debe mostrar temperatura, condiciones e ícono & Media \\ \hline
RF-WEAT-04 & El sistema debe permitir listar historial de búsquedas & Baja \\ \hline
\end{tabularx}
\end{table}

\subsection{Módulo de Administración}

\begin{table}[H]
\centering
\caption{Requisitos Funcionales - Admin}
\label{tab:req_admin}
\small
\begin{tabularx}{\textwidth}{|l|X|c|}
\hline
\textbf{ID} & \textbf{Requisito} & \textbf{Prioridad} \\ \hline
RF-ADM-01 & El sistema debe permitir a ADMIN listar todos los usuarios & Alta \\ \hline
RF-ADM-02 & El sistema debe permitir a ADMIN cambiar roles de usuarios & Alta \\ \hline
RF-ADM-03 & El sistema debe impedir a usuarios regulares acceder a rutas admin & Alta \\ \hline
\end{tabularx}
\end{table}

% ============================================================
% 3. MATRIZ DE RASTREABILIDAD
% ============================================================
\section{Matriz de Rastreabilidad: Requisitos vs Casos de Prueba}

\begin{landscape}
\begin{longtable}{|l|l|p{6cm}|l|l|l|}
\caption{Matriz de Rastreabilidad - Requisitos vs Casos de Prueba} \label{tab:rastreabilidad} \\
\hline
\textbf{ID Req.} & \textbf{ID Caso} & \textbf{Nombre del Caso de Prueba} & \textbf{Nivel} & \textbf{Herramienta} & \textbf{Estado Ejecución} \\ \hline
\endfirsthead

\multicolumn{6}{c}{\tablename\ \thetable\ -- Continuación} \\
\hline
\textbf{ID Req.} & \textbf{ID Caso} & \textbf{Nombre del Caso de Prueba} & \textbf{Nivel} & \textbf{Herramienta} & \textbf{Estado Ejecución} \\ \hline
\endhead

\hline \multicolumn{6}{r}{Continúa en la siguiente página...} \\
\endfoot

\hline
\endlastfoot

% AUTENTICACIÓN
RF-AUTH-01 & TC-AUTH-001 & Registro exitoso con datos válidos & E2E & Cypress & \estadopending \\ \hline
RF-AUTH-01 & TC-AUTH-002 & Registro falla con email duplicado & E2E & Cypress & \estadopending \\ \hline
RF-AUTH-02 & TC-AUTH-003 & Registro rechaza email inválido & Integración & Postman & \estadopending \\ \hline
RF-AUTH-03 & TC-AUTH-004 & Registro rechaza password corto & Integración & Postman & \estadopending \\ \hline
RF-AUTH-04 & TC-AUTH-U01 & bcrypt hashea contraseñas correctamente & Unitaria & Jest & \estadowarning \\ \hline
RF-AUTH-05 & TC-AUTH-005 & Login exitoso con credenciales válidas & E2E & Cypress & \\ \hline
RF-AUTH-06 & TC-AUTH-U02 & JWT generado contiene userId y expira en 24h & Unitaria & Jest & \estadowarning \\ \hline
RF-AUTH-07 & TC-AUTH-006 & Login fallido con credenciales inválidas & E2E & Cypress & \\ \hline
RF-AUTH-08 & TC-AUTH-007 & Logout exitoso elimina token del cliente & E2E & Cypress & \estadopending \\ \hline

% BIOMETRÍA
RF-BIO-01 & TC-BIO-001 & Registro biométrico exitoso con imagen válida & E2E & Cypress & \estadopending \\ \hline
RF-BIO-02 & TC-BIO-002 & Challenge token generado expira en 2 minutos & Integración & Postman & \estadopending \\ \hline
RF-BIO-03 & TC-BIO-003 & Microservicio extrae encoding 128D correctamente & Integración & Postman & \estadopending \\ \hline
RF-BIO-04 & TC-BIO-U01 & Encoding cifrado con AES-256 puede descifrarse & Unitaria & Jest & \estadowarning \\ \hline
RF-BIO-05 & TC-BIO-004 & Verificación rechaza imagen con liveness bajo & Integración & Postman & \estadopending \\ \hline
RF-BIO-06 & TC-BIO-005 & Verificación rechaza imagen de baja calidad & Integración & Postman & \estadopending \\ \hline
RF-BIO-07 & TC-BIO-006 & Login biométrico exitoso con rostro registrado & E2E & Cypress & \estadopending \\ \hline
RF-BIO-08 & TC-BIO-U02 & Auditoría registra timestamp e IP del intento & Unitaria & Jest & \estadowarning \\ \hline

% VIAJES
RF-TRIP-01 & TC-TRIP-001 & Crear viaje exitosamente con datos completos & E2E & Cypress &  \\ \hline
RF-TRIP-02 & TC-TRIP-002 & Creación rechaza startDate posterior a endDate & Integración & Postman & \estadopending \\ \hline
RF-TRIP-03 & TC-TRIP-003 & Listar viajes retorna solo viajes propios & Integración & Postman & \estadopending \\ \hline
RF-TRIP-04 & TC-TRIP-004 & Actualizar viaje propio exitosamente & E2E & Cypress & \estadopending \\ \hline
RF-TRIP-05 & TC-TRIP-005 & Eliminar viaje propio exitosamente & E2E & Cypress & \estadopending \\ \hline
RF-TRIP-06 & TC-TRIP-006 & Acceso a viaje ajeno retorna 403 Forbidden & Integración & Postman & \estadopending \\ \hline
RF-TRIP-06 & TC-SEC-001 & Prevención de IDOR en endpoint /trips/:id & Seguridad & OWASP ZAP & \estadopending \\ \hline

% CLIMA
RF-WEAT-01 & TC-WEAT-001 & Búsqueda de clima retorna datos válidos & Integración & Postman & \estadopending \\ \hline
RF-WEAT-02 & TC-WEAT-002 & Búsqueda almacena registro en BD & Unitaria & Jest & \estadowarning \\ \hline
RF-WEAT-03 & TC-WEAT-003 & Respuesta incluye temp, description e icon & Integración & Postman & \estadopending \\ \hline
RF-WEAT-04 & TC-WEAT-004 & Listar historial retorna búsquedas ordenadas & E2E & Cypress & \estadopending \\ \hline

% ADMINISTRACIÓN
RF-ADM-01 & TC-ADM-001 & Admin lista todos los usuarios & Integración & Postman & \estadopending \\ \hline
RF-ADM-02 & TC-ADM-002 & Admin cambia rol de usuario exitosamente & Integración & Postman & \estadopending \\ \hline
RF-ADM-03 & TC-ADM-003 & Usuario regular no accede a /api/users & Integración & Postman & \estadopending \\ \hline
RF-ADM-03 & TC-SEC-002 & Protección de rutas admin sin bypass & Seguridad & OWASP ZAP & \estadopending \\ \hline

% SEGURIDAD GENERAL
- & TC-SEC-003 & Detección de vulnerabilidades XSS en formularios & Seguridad & OWASP ZAP & \estadopending \\ \hline
- & TC-SEC-004 & Detección de SQL Injection en parámetros & Seguridad & OWASP ZAP & \estadopending \\ \hline
- & TC-SEC-005 & Headers de seguridad presentes (CSP, HSTS) & Seguridad & OWASP ZAP & \estadopending \\ \hline
- & TC-SEC-006 & Sin exposición de información sensible en errores & Seguridad & OWASP ZAP & \estadopending \\ \hline
- & TC-SEC-007 & JWT no puede ser manipulado sin invalidar firma & Seguridad & Manual & \estadopending \\ \hline

% PRUEBAS REALMENTE EJECUTADAS (adicionales)
- & TC-LAND-001 & Carga la página principal & E2E & Cypress &  \\ \hline
- & TC-NAV-001 & Navegar al dashboard después del login & E2E & Cypress &  \\ \hline
- & TC-NAV-002 & Acceder a la página de destinos & E2E & Cypress & \\ \hline
- & TC-NAV-003 & Acceder a la página del clima & E2E & Cypress & \\ \hline
- & TC-NAV-004 & Acceder a la página de perfil & E2E & Cypress & \\ \hline
- & TC-SEC-002A & Verificar protección de rutas sin autenticación & E2E & Cypress & \\ \hline

\end{longtable}
\end{landscape}

% ============================================================
% 4. DISEÑO DE CASOS DE PRUEBA UNITARIAS
% ============================================================
\section{Diseño de Casos de Prueba Unitarias}

\subsection{Casos de Prueba Unitaria - Backend (Jest)}

\subsubsection{TC-AUTH-U01: Hashing de Contraseñas con bcrypt}

\begin{table}[H]
\centering
\caption{TC-AUTH-U01: Hashing de Contraseñas}
\label{tab:tc_auth_u01}
\begin{tabularx}{\textwidth}{|l|>{\raggedright\arraybackslash}X|}
\hline
\textbf{ID} & TC-AUTH-U01 \\ \hline
\textbf{Nombre} & Verificar hashing correcto de contraseñas con bcrypt \\ \hline
\textbf{Módulo} & authController.js - register() \\ \hline
\textbf{Prioridad} & Alta \\ \hline
\textbf{Precondiciones} & Módulo bcrypt importado. Función register() disponible. \\ \hline
\textbf{Datos de Entrada} & password: "Test123!" \\ \hline
\textbf{Pasos} & 
1. Mock de User.findOne() retornando null\newline
2. Mock de bcrypt.hash() retornando hash predeterminado\newline
3. Llamar a register() con datos de prueba\newline
4. Verificar que bcrypt.hash fue llamado con password \\ \hline
\textbf{Resultado Esperado} & 
1. bcrypt.hash() llamado exactamente 1 vez\newline
2. Password no almacenado en texto plano\newline
3. Hash almacenado en BD es string de 60 caracteres \\ \hline
\textbf{Resultado Obtenido} & \estadowarning \textbf{No ejecutado exitosamente} - Suite de pruebas fallida \\ \hline
\textbf{Estado} & Fallido - Requiere corrección \\ \hline
\end{tabularx}
\end{table}

\textbf{Script de Prueba:}

\begin{lstlisting}[language=JavaScript, caption={TC-AUTH-U01: Jest Script}]
const { register } = require("../src/controllers/authController");
const User = require("../src/models/User");
const bcrypt = require("bcrypt");

jest.mock("../src/models/User");
jest.mock("bcrypt");

describe("TC-AUTH-U01: Password Hashing with bcrypt", () => {
let req, res;

beforeEach(() => {
    req = {
    body: {
        email: "test@mail.com",
        password: "Test123!",
        username: "testuser"
    }
    };
    res = {
    status: jest.fn().mockReturnThis(),
    json: jest.fn()
    };
});

afterEach(() => {
    jest.clearAllMocks();
});

it("should hash password with bcrypt before storing", async () => {
    // Arrange
    const hashedPassword = "$2b$10$abcdefghijklmnopqrstuvwxyz123456";
    User.findOne.mockResolvedValue(null);
    bcrypt.hash.mockResolvedValue(hashedPassword);
    User.prototype.save = jest.fn().mockResolvedValue({
    _id: "123",
    email: "test@mail.com",
    password: hashedPassword
    });

    // Act
    await register(req, res);

    // Assert
    expect(bcrypt.hash).toHaveBeenCalledTimes(1);
    expect(bcrypt.hash).toHaveBeenCalledWith("Test123!", 10);
    expect(User.prototype.save).toHaveBeenCalled();
    const savedUser = await User.prototype.save.mock.results[0].value;
    expect(savedUser.password).toBe(hashedPassword);
    expect(savedUser.password).toHaveLength(60);
});

it("should not store password in plain text", async () => {
    User.findOne.mockResolvedValue(null);
    bcrypt.hash.mockResolvedValue("hashedPassword");
    User.prototype.save = jest.fn().mockResolvedValue({
    password: "hashedPassword"
    });

    await register(req, res);

    const savedUser = await User.prototype.save.mock.results[0].value;
    expect(savedUser.password).not.toBe("Test123!");
});
});
\end{lstlisting}

\subsubsection{TC-AUTH-U02: Generación y Expiración de JWT}

\begin{table}[H]
\centering
\caption{TC-AUTH-U02: Generación de JWT}
\label{tab:tc_auth_u02}
\begin{tabularx}{\textwidth}{|l|>{\raggedright\arraybackslash}X|}
\hline
\textbf{ID} & TC-AUTH-U02 \\ \hline
\textbf{Nombre} & Verificar generación correcta de JWT con userId y expiración 24h \\ \hline
\textbf{Módulo} & authController.js - login() \\ \hline
\textbf{Prioridad} & Alta \\ \hline
\textbf{Precondiciones} & Módulo jsonwebtoken importado. JWT\_SECRET configurado. \\ \hline
\textbf{Datos de Entrada} & userId: "507f1f77bcf86cd799439011" \\ \hline
\textbf{Pasos} & 
1. Mock de User.findOne() retornando usuario válido\newline
2. Mock de bcrypt.compare() retornando true\newline
3. Llamar a login() con credenciales válidas\newline
4. Decodificar JWT retornado\newline
5. Verificar payload y expiración \\ \hline
\textbf{Resultado Esperado} & 
1. JWT contiene userId en payload\newline
2. exp (expiration) = iat + 24 horas (86400 seg)\newline
3. JWT firma válida con JWT\_SECRET \\ \hline
\textbf{Resultado Obtenido} & \estadowarning \textbf{No ejecutado exitosamente} - Parte de suite fallida \\ \hline
\textbf{Estado} & Fallido - Requiere corrección \\ \hline
\end{tabularx}
\end{table}

\textbf{Script de Prueba:}

\begin{lstlisting}[language=JavaScript, caption={TC-AUTH-U02: Jest Script}]
const { login } = require("../src/controllers/authController");
const User = require("../src/models/User");
const bcrypt = require("bcrypt");
const jwt = require("jsonwebtoken");

jest.mock("../src/models/User");
jest.mock("bcrypt");

describe("TC-AUTH-U02: JWT Generation and Expiration", () => {
const JWT_SECRET = "test-secret-key";
process.env.JWT_SECRET = JWT_SECRET;

let req, res;

beforeEach(() => {
    req = {
    body: {
        email: "test@mail.com",
        password: "Test123!"
    }
    };
    res = {
    status: jest.fn().mockReturnThis(),
    json: jest.fn()
    };
});

it("should generate JWT with userId and 24h expiration", async () => {
    // Arrange
    const mockUser = {
    _id: "507f1f77bcf86cd799439011",
    email: "test@mail.com",
    password: "hashedPassword",
    username: "testuser"
    };
    User.findOne.mockResolvedValue(mockUser);
    bcrypt.compare.mockResolvedValue(true);

    // Act
    await login(req, res);

    // Assert
    expect(res.status).toHaveBeenCalledWith(200);
    const response = res.json.mock.calls[0][0];
    const token = response.data.token;

    // Decode JWT
    const decoded = jwt.verify(token, JWT_SECRET);
    
    expect(decoded.userId).toBe(mockUser._id);
    
    // Verify 24h expiration (86400 seconds)
    const expirationTime = decoded.exp - decoded.iat;
    expect(expirationTime).toBe(86400);
});

it("should sign JWT with JWT_SECRET", async () => {
    const mockUser = {
    _id: "123",
    email: "test@mail.com",
    password: "hashedPassword"
    };
    User.findOne.mockResolvedValue(mockUser);
    bcrypt.compare.mockResolvedValue(true);

    await login(req, res);

    const response = res.json.mock.calls[0][0];
    const token = response.data.token;

    // Should not throw error
    expect(() => {
    jwt.verify(token, JWT_SECRET);
    }).not.toThrow();

    // Should throw with wrong secret
    expect(() => {
    jwt.verify(token, "wrong-secret");
    }).toThrow();
});
});
\end{lstlisting}

\subsection{Casos de Prueba Unitaria - Frontend (Jest + RTL)}

\subsubsection{TC-FE-U01: Componente BiometricLogin Renderiza Correctamente}

\begin{table}[H]
\centering
\caption{TC-FE-U01: Renderizado de BiometricLogin}
\label{tab:tc_fe_u01}
\begin{tabular}{|l|p{\dimexpr\textwidth-3cm}|}
\hline
\textbf{ID} & TC-FE-U01 \\ \hline
\textbf{Nombre} & Verificar renderizado correcto de BiometricLogin component \\ \hline
\textbf{Componente} & BiometricLoginAdvanced.jsx \\ \hline
\textbf{Prioridad} & Media \\ \hline
\textbf{Precondiciones} & React Testing Library configurado. face-api.js mockeado. \\ \hline
\textbf{Pasos} & 
1. Renderizar componente BiometricLoginAdvanced\newline
2. Buscar elementos clave del DOM\newline
3. Verificar visibilidad de botones \\ \hline
\textbf{Resultado Esperado} & 
1. Input de email visible\newline
2. Botón "Start Face Detection" visible\newline
3. Canvas de video presente en DOM \\ \hline
\textbf{Resultado Obtenido} & \textbf{Ejecutado exitosamente} - Parte de pruebas frontend completadas \\ \hline
\textbf{Estado} & Completado \\ \hline
\end{tabular}
\end{table}

\textbf{Script de Prueba:}

\begin{lstlisting}[language=JavaScript, caption={TC-FE-U01: RTL Script}]
import { render, screen } from "@testing-library/react";
import { BrowserRouter } from "react-router-dom";
import BiometricLoginAdvanced from "../src/components/BiometricLoginAdvanced";

// Mock face-api.js
jest.mock("face-api.js", () => ({
nets: {
    tinyFaceDetector: { loadFromUri: jest.fn() },
    faceLandmark68Net: { loadFromUri: jest.fn() },
    faceExpressionNet: { loadFromUri: jest.fn() }
},
detectSingleFace: jest.fn(),
TinyFaceDetectorOptions: jest.fn()
}));

describe("TC-FE-U01: BiometricLogin Component Rendering", () => {
it("should render email input field", () => {
    render(
    <BrowserRouter>
        <BiometricLoginAdvanced />
    </BrowserRouter>
    );

    const emailInput = screen.getByLabelText(/email/i);
    expect(emailInput).toBeInTheDocument();
    expect(emailInput).toHaveAttribute("type", "email");
});

it("should render start detection button", () => {
    render(
    <BrowserRouter>
        <BiometricLoginAdvanced />
    </BrowserRouter>
    );

    const startButton = screen.getByText(/start face detection/i);
    expect(startButton).toBeInTheDocument();
    expect(startButton).toBeEnabled();
});

it("should render video canvas element", () => {
    const { container } = render(
    <BrowserRouter>
        <BiometricLoginAdvanced />
    </BrowserRouter>
    );

    const canvas = container.querySelector("canvas");
    expect(canvas).toBeInTheDocument();
});
});
\end{lstlisting}

% ============================================================
% 5. DISEÑO DE CASOS DE PRUEBA DE INTEGRACIÓN
% ============================================================
\section{Diseño de Casos de Prueba de Integración}

\subsection{Casos de Prueba de Integración - APIs (Postman)}

\subsubsection{TC-AUTH-003: Validación de Email Inválido}

\begin{table}[H]
\centering
\caption{TC-AUTH-003: Email Inválido}
\label{tab:tc_auth_003}
\begin{tabularx}{\textwidth}{|l|>{\raggedright\arraybackslash}X|}
\hline
\textbf{ID} & TC-AUTH-003 \\ \hline
\textbf{Nombre} & Registro rechaza email con formato inválido \\ \hline
\textbf{Endpoint} & POST /api/auth/register \\ \hline
\textbf{Prioridad} & Alta \\ \hline
\textbf{Precondiciones} & Backend desplegado en http://localhost:4000. Base de datos accesible. \\ \hline
\textbf{Headers} & Content-Type: application/json \\ \hline
\textbf{Body (JSON)} & 
\{\newline
"email": "invalid-email-format",\newline
"password": "Test123!",\newline
"username": "testuser"\newline
\} \\ \hline
\textbf{Resultado Esperado} & 
1. Status Code: 400 Bad Request\newline
2. Response: \{"success": false, "message": "Invalid email format"\}\newline
3. No se crea usuario en BD \\ \hline
\textbf{Test Script (Postman)} & 
pm.test("Status is 400", () => \{\newline
pm.response.to.have.status(400);\newline
\});\newline
\newline
pm.test("Error message present", () => \{\newline
const json = pm.response.json();\newline
pm.expect(json.success).to.be.false;\newline
pm.expect(json.message).to.include("email");\newline
\}); \\ \hline
\textbf{Resultado Obtenido} & \estadopending \textbf{Pendiente de ejecución} \\ \hline
\textbf{Estado} & No Ejecutado \\ \hline
\end{tabularx}
\end{table}

\subsubsection{TC-BIO-003: Extracción de Encoding 128D}

\begin{table}[H]
\centering
\caption{TC-BIO-003: Extracción de Encoding}
\label{tab:tc_bio_003}
\begin{tabularx}{\textwidth}{|l|>{\raggedright\arraybackslash}X|}
\hline
\textbf{ID} & TC-BIO-003 \\ \hline
\textbf{Nombre} & Microservicio extrae encoding 128D correctamente de imagen facial \\ \hline
\textbf{Endpoint} & POST http://localhost:8001/extract-features (Interno) \\ \hline
\textbf{Prioridad} & Alta \\ \hline
\textbf{Precondiciones} & 
Facial service desplegado. Backend tiene acceso a red interna. INTERNAL\_SERVICE\_TOKEN configurado. \\ \hline
\textbf{Headers} & X-Internal-Token: \{\{internal\_service\_token\}\} \\ \hline
\textbf{Body (multipart)} & face: [archivo face.jpg válido] \\ \hline
\textbf{Resultado Esperado} & 
1. Status Code: 200 OK\newline
2. Response contiene encoding: array de 128 floats\newline
3. liveness score: 0.0-1.0\newline
4. quality score: 0.0-1.0\newline
5. confidence: 0.0-1.0 \\ \hline
\textbf{Test Script} & 
pm.test("Status is 200", () => \{\newline
pm.response.to.have.status(200);\newline
\});\newline
\newline
pm.test("Encoding is 128D array", () => \{\newline
const json = pm.response.json();\newline
pm.expect(json.encoding).to.be.an("array");\newline
pm.expect(json.encoding).to.have.lengthOf(128);\newline
\});\newline
\newline
pm.test("Liveness score present", () => \{\newline
const json = pm.response.json();\newline
pm.expect(json.liveness\_score).to.be.a("number");\newline
pm.expect(json.liveness\_score).to.be.within(0, 1);\newline
\}); \\ \hline
\textbf{Resultado Obtenido} & \estadopending \textbf{Pendiente de ejecución} \\ \hline
\textbf{Estado} & No Ejecutado \\ \hline
\end{tabularx}
\end{table}

\subsubsection{TC-TRIP-006: Protección IDOR en Viajes}

\begin{table}[H]
\centering
\caption{TC-TRIP-006: Protección IDOR}
\label{tab:tc_trip_006}
\begin{tabularx}{\textwidth}{|l|>{\raggedright\arraybackslash}X|}
\hline
\textbf{ID} & TC-TRIP-006 \\ \hline
\textbf{Nombre} & Usuario no puede acceder a viajes de otros usuarios (IDOR prevention) \\ \hline
\textbf{Endpoint} & GET /trips/:tripId \\ \hline
\textbf{Prioridad} & Alta \\ \hline
\textbf{Precondiciones} & 
Usuario A autenticado (JWT token en variable). Usuario B tiene viaje con \_id conocido. \\ \hline
\textbf{Headers} & Authorization: Bearer \{\{jwt\_token\_userA\}\} \\ \hline
\textbf{URL} & GET http://localhost:4000/trips/\{\{trip\_id\_userB\}\} \\ \hline
\textbf{Resultado Esperado} & 
1. Status Code: 403 Forbidden\newline
2. Response: \{"success": false, "message": "Access denied"\}\newline
3. No se retornan datos del viaje ajeno \\ \hline
\textbf{Test Script} & 
pm.test("Status is 403", () => \{\newline
pm.response.to.have.status(403);\newline
\});\newline
\newline
pm.test("Access denied message", () => \{\newline
const json = pm.response.json();\newline
pm.expect(json.success).to.be.false;\newline
pm.expect(json.message).to.include("denied");\newline
\}); \\ \hline
\textbf{Resultado Obtenido} & \estadopending \textbf{Pendiente de ejecución} \\ \hline
\textbf{Estado} & No Ejecutado \\ \hline
\end{tabularx}
\end{table}

% ============================================================
% 6. DISEÑO DE CASOS DE PRUEBA E2E
% ============================================================
\section{Diseño de Casos de Prueba End-to-End}

\subsection{Casos de Prueba E2E - Frontend (Cypress)}

\subsubsection{TC-AUTH-005: Login Tradicional Exitoso}

\begin{table}[H]
\centering
\caption{TC-AUTH-005: Login Exitoso}
\label{tab:tc_auth_005}
\begin{tabularx}{\textwidth}{|l|>{\raggedright\arraybackslash}X|}
\hline
\textbf{ID} & TC-AUTH-005 \\ \hline
\textbf{Nombre} & Login tradicional exitoso con credenciales válidas \\ \hline
\textbf{Flujo} & Autenticación \\ \hline
\textbf{Prioridad} & Alta \\ \hline
\textbf{Precondiciones} & 
Usuario registrado: test@test.com / admin1234. Frontend accesible en http://localhost:3001. Backend funcional. \\ \hline
\textbf{Pasos} & 
1. Navegar a http://travelbrain.ddns.net/login\newline
2. Ingresar email: test@test.com\newline
3. Ingresar password: admin1234\newline
4. Hacer clic en botón "Sign in"\newline
5. Esperar redirección \\ \hline
\textbf{Resultado Esperado} & 
1. Redirección a /dashboard\newline
2. Mensaje de bienvenida visible\newline
3. Token JWT almacenado en localStorage\newline
4. Navbar muestra nombre del usuario \\ \hline
\textbf{Resultado Obtenido} & \estadook \textbf{Ejecutado exitosamente} \\ \hline
\textbf{Estado} & Completado \\ \hline
\end{tabularx}
\end{table}

\textbf{Script Cypress Ejecutado:}

\begin{lstlisting}[language=JavaScript, caption={TC-AUTH-005: Cypress Script Ejecutado}]
describe('Login de usuario', () => {
  it('Login exitoso con credenciales válidas', () => {
    cy.visit('https://travelbrain.ddns.net/login')
    
    // Ingresar credenciales válidas
    cy.get('input[name="email"]').type('test@test.com')
    cy.get('input[type="password"]').type('admin1234')
    
    // Hacer clic en el botón de login
    cy.get('button[type="submit"]').click()
    
    // Verificar redirección al dashboard
    cy.url().should('include', '/dashboard')
    
    // Verificar elementos del dashboard
    cy.contains('Welcome back').should('be.visible')
    cy.contains('Total Trips').should('be.visible')
  })
})
\end{lstlisting}

\subsubsection{TC-AUTH-006: Login Fallido con Credenciales Inválidas}

\begin{table}[H]
\centering
\caption{TC-AUTH-006: Login Fallido}
\label{tab:tc_auth_006}
\begin{tabularx}{\textwidth}{|l|>{\raggedright\arraybackslash}X|}
\hline
\textbf{ID} & TC-AUTH-006 \\ \hline
\textbf{Nombre} & Login fallido con credenciales inválidas \\ \hline
\textbf{Flujo} & Autenticación \\ \hline
\textbf{Prioridad} & Alta \\ \hline
\textbf{Precondiciones} & Frontend accesible. Usuario no registrado. \\ \hline
\textbf{Pasos} & 
1. Navegar a https://travelbrain.ddns.net/login\newline
2. Ingresar email: ihopc@gmail.com\newline
3. Ingresar password: incorrecta123\newline
4. Hacer clic en botón "Sign in"\newline
5. Esperar respuesta \\ \hline
\textbf{Resultado Esperado} & 
1. Permanece en página /login\newline
2. Mensaje de error visible\newline
3. No se redirige al dashboard\newline
4. Token no almacenado \\ \hline
\textbf{Resultado Obtenido} & \estadook \textbf{Ejecutado exitosamente} - Mensaje "Login failed. Please try again." \\ \hline
\textbf{Estado} & Completado \\ \hline
\end{tabularx}
\end{table}

\subsubsection{TC-TRIP-001: Crear Viaje Exitosamente}

\begin{table}[H]
\centering
\caption{TC-TRIP-001: Crear Viaje}
\label{tab:tc_trip_001}
\begin{tabularx}{\textwidth}{|l|>{\raggedright\arraybackslash}X|}
\hline
\textbf{ID} & TC-TRIP-001 \\ \hline
\textbf{Nombre} & Crear viaje exitosamente con datos completos \\ \hline
\textbf{Flujo} & Gestión de Viajes \\ \hline
\textbf{Prioridad} & Alta \\ \hline
\textbf{Precondiciones} & Usuario autenticado. En página /trips. \\ \hline
\textbf{Datos de Prueba} & 
Título: "Viaje E2E Cypress". Destino: "Paris, France". Fecha inicio: 2026-06-01. Fecha fin: 2026-06-15. Presupuesto: 5000. \\ \hline
\textbf{Pasos} & 
1. Login con credenciales válidas\newline
2. Navegar a /trips\newline
3. Hacer clic en "New Trip"\newline
4. Completar formulario con datos de prueba\newline
5. Hacer clic en "Save"\newline
6. Esperar respuesta \\ \hline
\textbf{Resultado Esperado} & 
1. Mensaje de éxito visible\newline
2. Viaje aparece en lista\newline
3. Datos coinciden con ingresados \\ \hline
\textbf{Resultado Obtenido} & \estadook \textbf{Ejecutado exitosamente} \\ \hline
\textbf{Estado} & Completado \\ \hline
\end{tabularx}
\end{table}

\textbf{Script Cypress Ejecutado:}

\begin{lstlisting}[language=JavaScript, caption={TC-TRIP-001: Cypress Script Ejecutado}]
describe('Crear un viaje', () => {
  it('Crear viaje exitosamente', () => {
    // Login primero
    cy.visit('https://travelbrain.ddns.net/login')
    cy.get('input[name="email"]').type('test@test.com')
    cy.get('input[type="password"]').type('admin1234')
    cy.get('button[type="submit"]').click()
    
    // Navegar a trips
    cy.visit('https://travelbrain.ddns.net/trips')
    
    // Abrir modal de crear viaje
    cy.contains('New Trip').click()
    
    // Completar formulario
    cy.get('input[name="title"]').type('Viaje E2E Cypress')
    cy.get('input[name="destination"]').type('Paris, France')
    cy.get('input[name="startDate"]').type('2026-06-01')
    cy.get('input[name="endDate"]').type('2026-06-15')
    cy.get('input[name="budget"]').type('5000')
    
    // Enviar formulario
    cy.get('button[type="submit"]').click()
    
    // Verificar éxito
    cy.contains('Trip created successfully').should('be.visible')
    cy.contains('Viaje E2E Cypress').should('be.visible')
  })
})
\end{lstlisting}

\subsubsection{TC-LAND-001: Carga de Página Principal}

\begin{table}[H]
\centering
\caption{TC-LAND-001: Carga Página Principal}
\label{tab:tc_land_001}
\begin{tabularx}{\textwidth}{|l|>{\raggedright\arraybackslash}X|}
\hline
\textbf{ID} & TC-LAND-001 \\ \hline
\textbf{Nombre} & Carga exitosa de la página principal \\ \hline
\textbf{Flujo} & Landing \\ \hline
\textbf{Prioridad} & Alta \\ \hline
\textbf{Precondiciones} & Servidor web funcionando. \\ \hline
\textbf{Pasos} & 
1. Navegar a https://travelbrain.ddns.net/\newline
2. Esperar carga completa \\ \hline
\textbf{Resultado Esperado} & 
1. Página carga sin errores\newline
2. Texto "TRAVELBRAIN" visible\newline
3. Botones de login/registro presentes \\ \hline
\textbf{Resultado Obtenido} & \estadook \textbf{Ejecutado exitosamente} \\ \hline
\textbf{Estado} & Completado \\ \hline
\end{tabularx}
\end{table}

% ============================================================
% 7. DISEÑO DE CASOS DE PRUEBA DE SEGURIDAD
% ============================================================
\section{Diseño de Casos de Prueba de Seguridad}

\subsection{Casos de Prueba de Seguridad - OWASP ZAP}

\subsubsection{TC-SEC-001: Protección IDOR}

\begin{table}[H]
\centering
\caption{TC-SEC-001: IDOR Prevention}
\label{tab:tc_sec_001}
\begin{tabularx}{\textwidth}{|l|>{\raggedright\arraybackslash}X|}
\hline
\textbf{ID} & TC-SEC-001 \\ \hline
\textbf{Nombre} & Prevención de IDOR (Insecure Direct Object Reference) en /trips/:id \\ \hline
\textbf{Vulnerabilidad} & OWASP A01:2021 - Broken Access Control \\ \hline
\textbf{Prioridad} & Alta \\ \hline
\textbf{Método} & Escaneo Activo + Manual \\ \hline
\textbf{Precondiciones} & 
OWASP ZAP configurado como proxy. 2 usuarios con sesiones activas (UserA, UserB). UserB tiene viaje con \_id conocido. \\ \hline
\textbf{Pasos} & 
1. Autenticar como UserA en ZAP\newline
2. Interceptar GET /trips/:id con token de UserA\newline
3. Modificar :id por ID de viaje de UserB\newline
4. Enviar request\newline
5. Analizar respuesta \\ \hline
\textbf{Resultado Esperado} & 
1. HTTP 403 Forbidden\newline
2. No se retornan datos del viaje ajeno\newline
3. No hay information leakage \\ \hline
\textbf{Criterio de Éxito} & Sin vulnerabilidad IDOR detectada \\ \hline
\textbf{Resultado Obtenido} & \estadopending \textbf{Pendiente de ejecución} \\ \hline
\textbf{Estado} & No Ejecutado \\ \hline
\end{tabularx}
\end{table}

\subsubsection{TC-SEC-003: Cross-Site Scripting (XSS)}

\begin{table}[H]
\centering
\caption{TC-SEC-003: XSS Detection}
\label{tab:tc_sec_003}
\begin{tabularx}{\textwidth}{|l|>{\raggedright\arraybackslash}X|}
\hline
\textbf{ID} & TC-SEC-003 \\ \hline
\textbf{Nombre} & Detección de vulnerabilidades XSS en formularios de entrada \\ \hline
\textbf{Vulnerabilidad} & OWASP A03:2021 - Injection \\ \hline
\textbf{Prioridad} & Alta \\ \hline
\textbf{Método} & Escaneo Activo (ZAP Spider + Active Scan) \\ \hline
\textbf{Targets} & 
/register (username, email). /trips (title, description, destination). /weather (city search). \\ \hline
\textbf{Payloads} & 
\texttt{<script>alert("XSS")</script>}\newline
\texttt{<img src=x onerror=alert("XSS")>}\newline
\texttt{"><script>alert(String.fromCharCode(88,83,83))</script>} \\ \hline
\textbf{Pasos} & 
1. Configurar contexto en ZAP\newline
2. Ejecutar Spider en http://localhost:3001\newline
3. Ejecutar Active Scan con XSS rules\newline
4. Revisar alertas generadas\newline
5. Verificar manualmente payloads sospechosos \\ \hline
\textbf{Resultado Esperado} & 
1. 0 alertas de XSS de severidad Alta/Crítica\newline
2. Inputs sanitizados correctamente\newline
3. React escapa HTML automáticamente \\ \hline
\textbf{Criterio de Éxito} & Sin XSS ejecutable detectado \\ \hline
\textbf{Resultado Obtenido} & \estadopending \textbf{Pendiente de ejecución} \\ \hline
\textbf{Estado} & No Ejecutado \\ \hline
\end{tabularx}
\end{table}

\subsubsection{TC-SEC-005: Headers de Seguridad}

\begin{table}[H]
\centering
\caption{TC-SEC-005: Security Headers}
\label{tab:tc_sec_005}
\begin{tabularx}{\textwidth}{|l|>{\raggedright\arraybackslash}X|}
\hline
\textbf{ID} & TC-SEC-005 \\ \hline
\textbf{Nombre} & Verificación de headers de seguridad HTTP \\ \hline
\textbf{Categoría} & Security Misconfiguration \\ \hline
\textbf{Prioridad} & Media \\ \hline
\textbf{Método} & Passive Scan (ZAP) \\ \hline
\textbf{Targets} & Todas las respuestas HTTP del backend \\ \hline
\textbf{Headers Requeridos} & 
Strict-Transport-Security. X-Content-Type-Options: nosniff. X-Frame-Options: DENY. Content-Security-Policy. X-XSS-Protection: 1; mode=block. \\ \hline
\textbf{Pasos} & 
1. Navegar aplicación con ZAP proxy activo\newline
2. ZAP registra headers automáticamente\newline
3. Revisar alertas de "Missing Security Headers"\newline
4. Verificar cada header manualmente \\ \hline
\textbf{Resultado Esperado} & 
1. Todos los headers requeridos presentes\newline
2. HSTS con maxAge >= 31536000\newline
3. CSP configurado correctamente \\ \hline
\textbf{Criterio de Éxito} & Sin alertas de headers faltantes \\ \hline
\textbf{Resultado Obtenido} & \estadopending \textbf{Pendiente de ejecución} \\ \hline
\textbf{Estado} & No Ejecutado \\ \hline
\end{tabularx}
\end{table}

% ============================================================
% 8. RESULTADOS DE EJECUCIÓN Y ANÁLISIS
% ============================================================
\section{Resultados de Ejecución y Análisis}

\subsection{Estado Actual de Ejecución}

\begin{table}[H]
\centering
\caption{Resumen de Ejecución por Nivel}
\label{tab:resumen_ejecucion}
\begin{tabularx}{\textwidth}{|X|c|c|c|c|}
\hline
\textbf{Nivel de Prueba} & \textbf{Diseñados} & \textbf{Ejecutados} & \textbf{Exitosa} & \textbf{Estado} \\ \hline
Unitarias (Backend) & 18 & 10+ & 0/10 & \estadowarning Requiere corrección \\ \hline
Unitarias (Frontend) & 15 & 21+ & 21/21 & \estadook Completado \\ \hline
Integración (APIs) & 28 & 0 & 0/28 & \estadopending Pendiente \\ \hline
E2E (Cypress) & 16 & 9 & 9/9 & \estadook Completado \\ \hline
Seguridad (ZAP) & 10 & 0 & 0/10 & \estadopending Pendiente \\ \hline
\textbf{TOTAL} & \textbf{87} & \textbf{40+} & \textbf{30/87} & \textbf{34.5\% completado} \\ \hline
\end{tabularx}
\end{table}

\subsection{Resultados de SonarQube}

\begin{table}[H]
\centering
\caption{Análisis de Calidad - SonarQube}
\label{tab:sonarqube_analisis}
\begin{tabularx}{\textwidth}{|X|c|c|c|}
\hline
\textbf{Métrica} & \textbf{Valor Actual} & \textbf{Calificación} & \textbf{Estado} \\ \hline
Cobertura de Código & 15.3\% & D (Deficiente) & \estadowarning \\ \hline
Issues de Seguridad & 1 & A (Aceptable) & \estadowarning \\ \hline
Issues de Confiabilidad & 94 & C (Crítico) & \estadofailed \\ \hline
Issues de Mantenibilidad & 286 & A (Aceptable) & \estadowarning \\ \hline
Duplicación de Código & 2.6\% & C (Crítico) & \estadowarning \\ \hline
Líneas de Código & 19,001 & - & - \\ \hline
Líneas por Cubrir & 3,700 & - & \estadowarning \\ \hline
\textbf{QUALITY GATE} & \textbf{PASSED (con advertencias)} & - & \estadowarning \\ \hline
\end{tabularx}
\end{table}

\subsection{Análisis de Cobertura por Módulo}

\begin{table}[H]
\centering
\caption{Cobertura por Módulo Backend}
\label{tab:coverage_backend}
\begin{tabularx}{\textwidth}{|X|c|c|c|}
\hline
\textbf{Módulo} & \textbf{Cobertura} & \textbf{Líneas Cubiertas} & \textbf{Estado} \\ \hline
Total Backend & 15.3\% & 3.7k/23k & \estadofailed \\ \hline
backend-project/src & 17.2\% & 3,064 líneas & \estadowarning \\ \hline
business-rules-backend/src & 7.5\% & 1,505 líneas & \estadofailed \\ \hline
frontend-react/src & 17.1\% & 14,432 líneas & \estadowarning \\ \hline
Controllers & 22.64\% & - & \estadowarning \\ \hline
Services & 1.12\% & - & \estadofailed \\ \hline
Middlewares & 24.29\% & - & \estadowarning \\ \hline
Models & 72\% & - & \estadook \\ \hline
Routes & 43.62\% & - & \estadowarning \\ \hline
Utils & 13.46\% & - & \estadofailed \\ \hline
\end{tabularx}
\end{table}

\subsection{Pruebas Unitarias Ejecutadas}

\subsubsection{Frontend React - Pruebas Completadas}

\begin{table}[H]
\centering
\caption{Componentes Frontend Probados}
\label{tab:frontend_tests}
\begin{tabularx}{\textwidth}{|X|c|c|}
\hline
\textbf{Componente/Servicio} & \textbf{Tests} & \textbf{Estado} \\ \hline
AuthSuccess & Múltiples & \estadook \\ \hline
Admin & Múltiples & \estadook \\ \hline
App & Múltiples & \estadook \\ \hline
CurrencySelector & Múltiples & \estadook \\ \hline
Dashboard & Múltiples & \estadook \\ \hline
GoogleLoginButton & Múltiples & \estadook \\ \hline
Itineraries & Múltiples & \estadook \\ \hline
Landing & Múltiples & \estadook \\ \hline
Login & Múltiples & \estadook \\ \hline
Navbar & Múltiples & \estadook \\ \hline
Profile & Múltiples & \estadook \\ \hline
Register & Múltiples & \estadook \\ \hline
Trips & Múltiples & \estadook \\ \hline
Weather & Múltiples & \estadook \\ \hline
api (servicio) & Múltiples & \estadook \\ \hline
config & Múltiples & \estadook \\ \hline
currencyService & Múltiples & \estadook \\ \hline
itineraryService & Múltiples & \estadook \\ \hline
tripService & Múltiples & \estadook \\ \hline
useAuth (hook) & Múltiples & \estadook \\ \hline
weatherService & Múltiples & \estadook \\ \hline
\end{tabularx}
\end{table}

\subsubsection{Backend - Pruebas Ejecutadas con Fallos}

\begin{table}[H]
\centering
\caption{Pruebas Backend Ejecutadas}
\label{tab:backend_tests}
\begin{tabularx}{\textwidth}{|X|c|c|}
\hline
\textbf{Archivo de Prueba} & \textbf{Resultado} & \textbf{Detalles} \\ \hline
auth.test.js & \estadofailed & Suite fallida \\ \hline
performance.test.js & \estadofailed & Suite fallida \\ \hline
security.test.js & \estadofailed & Suite fallida \\ \hline
trips.test.js & \estadofailed & Suite fallida \\ \hline
trips.coverage.test.js & \estadofailed & Suite fallida \\ \hline
users.test.js & \estadofailed & Suite fallida \\ \hline
destinations.test.js & \estadofailed & Suite fallida \\ \hline
weather.test.js & \estadofailed & Suite fallida \\ \hline
itineraries.test.js & \estadofailed & Suite fallida \\ \hline
favoriteRoutes.test.js & \estadofailed & Suite fallida \\ \hline
\textbf{TOTAL} & \textbf{10 failed, 10 total} & \textbf{13 failed, 13 total tests} \\ \hline
\end{tabularx}
\end{table}

\subsubsection{Business Rules - Pruebas Ejecutadas con Fallos}

\begin{table}[H]
\centering
\caption{Pruebas Business Rules Ejecutadas}
\label{tab:business_rules_tests}
\begin{tabularx}{\textwidth}{|X|c|c|}
\hline
\textbf{Archivo de Prueba} & \textbf{Resultado} & \textbf{Detalles} \\ \hline
businessRules.test.js & \estadofailed & Suite fallida \\ \hline
itinerary.test.js & \estadofailed & Suite fallida \\ \hline
performance.test.js & \estadofailed & Suite fallida \\ \hline
security.test.js & \estadofailed & Suite fallida \\ \hline
trip.coverage.test.js & \estadofailed & Suite fallida \\ \hline
\textbf{TOTAL} & \textbf{5 failed, 5 total} & \textbf{4 failed, 4 total tests} \\ \hline
\end{tabularx}
\end{table}

\subsection{Pruebas E2E Ejecutadas Exitosamente}

\begin{table}[H]
\centering
\caption{Pruebas E2E Cypress Completadas}
\label{tab:e2e_tests}
\begin{tabularx}{\textwidth}{|X|c|c|}
\hline
\textbf{Prueba E2E} & \textbf{Resultado} & \textbf{Descripción} \\ \hline
Carga la página principal & \estadook & Verifica texto "TRAVELBRAIN" \\ \hline
Login exitoso & \estadook & Credenciales: test@test.com / admin1234 \\ \hline
Login fallido & \estadook & Manejo de credenciales incorrectas \\ \hline
Crear un viaje & \estadook & Viaje a Paris con datos completos \\ \hline
Navegar al dashboard & \estadook & Acceso post-login \\ \hline
Acceder a destinos & \estadook & Navegación a página de destinos \\ \hline
Acceder al clima & \estadook & Navegación a página de clima \\ \hline
Acceder a perfil & \estadook & Navegación a página de perfil \\ \hline
Protección de rutas & \estadook & Redirección a /login sin auth \\ \hline
\end{tabularx}
\end{table}

% ============================================================
% 9. ANÁLISIS DE ISSUES Y PROBLEMAS DETECTADOS
% ============================================================
\section{Análisis de Issues y Problemas Detectados}

\subsection{Issues Críticos Identificados}

\begin{table}[H]
\centering
\caption{Issues Críticos por Categoría}
\label{tab:issues_criticos}
\begin{tabularx}{\textwidth}{|l|c|p{8cm}|}
\hline
\textbf{Categoría} & \textbf{Cantidad} & \textbf{Descripción} \\ \hline
Confiabilidad & 94 & Issues que pueden causar fallos en producción. Prioridad alta de corrección. \\ \hline
Mantenibilidad & 286 & Problemas que dificultan el mantenimiento del código. \\ \hline
Seguridad & 1 & Vulnerabilidades de seguridad detectadas. \\ \hline
Cobertura Insuficiente & - & Cobertura del 15.3\% está muy por debajo del estándar del 70\%. \\ \hline
Pruebas Fallidas & 17 suites & Todas las suites de backend y business rules fallan. \\ \hline
Duplicación & 2.6\% & Código duplicado que aumenta complejidad. \\ \hline
\end{tabularx}
\end{table}

\subsection{Áreas Problemáticas Específicas}

\subsubsection{Backend Project - Cobertura por Servicio}
\begin{itemize}
    \item \textbf{Services (1.12\%):} Muy baja cobertura en lógica de negocio
    \begin{itemize}
        \item destinationBusinessRules.js: 1.4\%
        \item itineraryBusinessRules.js: 1.53\%
        \item routeBusinessRules.js: 0.79\%
        \item tripBusinessRules.js: 1.05\%
        \item userBusinessRules.js: 1.13\%
    \end{itemize}
    
    \item \textbf{Controllers (22.64\%):} Cobertura insuficiente en controladores
    \item \textbf{Utils (13.46\%):} Funciones auxiliares poco probadas
    \item \textbf{Middlewares (24.29\%):} Módulos de seguridad con cobertura baja
\end{itemize}

\subsubsection{Problemas en Pruebas Unitarias}
\begin{enumerate}
    \item \textbf{Configuración de Jest:} Posibles problemas con mocks
    \item \textbf{Base de datos de prueba:} Configuración incorrecta
    \item \textbf{Imports:} Errores en rutas de módulos
    \item \textbf{Variables de entorno:} No configuradas para entorno de prueba
\end{enumerate}

\subsection{Recomendaciones de Corrección Inmediata}

\begin{table}[H]
\centering
\caption{Plan de Corrección de Issues}
\label{tab:plan_correccion}
\begin{tabularx}{\textwidth}{|l|p{8cm}|c|}
\hline
\textbf{Prioridad} & \textbf{Acción Correctiva} & \textbf{Estimado} \\ \hline
P1 (Crítico) & Corregir configuración de pruebas Jest en backend & 4 horas \\ \hline
P1 (Crítico) & Implementar pruebas para services (al menos 70\%) & 8 horas \\ \hline
P1 (Crítico) & Resolver issues de confiabilidad más graves & 6 horas \\ \hline
P2 (Alta) & Ejecutar pruebas de integración con Postman & 4 horas \\ \hline
P2 (Alta) & Ejecutar pruebas de seguridad con OWASP ZAP & 6 horas \\ \hline
P3 (Media) & Reducir duplicación de código & 4 horas \\ \hline
P3 (Media) & Mejorar cobertura de controllers y middlewares & 6 horas \\ \hline
\end{tabularx}
\end{table}

% ============================================================
% 10. MATRIZ RESUMEN DE CASOS DE PRUEBA
% ============================================================
\section{Matriz Resumen de Casos de Prueba}

\begin{table}[H]
\centering
\caption{Resumen de Casos de Prueba por Nivel}
\label{tab:resumen_casos}
\begin{tabularx}{\textwidth}{|X|c|c|c|c|c|}
\hline
\textbf{Nivel de Prueba} & \textbf{Total Casos} & \textbf{Ejecutados} & \textbf{Exitosa} & \textbf{Fallida} & \textbf{Pendiente} \\ \hline
Unitarias (Backend) & 18 & 10+ & 0 & 10+ & 8 \\ \hline
Unitarias (Frontend) & 15 & 21+ & 21+ & 0 & 0 \\ \hline
Integración (APIs) & 28 & 0 & 0 & 0 & 28 \\ \hline
E2E (Cypress) & 16 & 9 & 9 & 0 & 7 \\ \hline
Seguridad (ZAP) & 10 & 0 & 0 & 0 & 10 \\ \hline
\textbf{TOTAL} & \textbf{87} & \textbf{40+} & \textbf{30+} & \textbf{10+} & \textbf{53} \\ \hline
\end{tabularx}
\end{table}

\begin{table}[H]
\centering
\caption{Distribución por Módulo con Estado Actual}
\label{tab:casos_modulo}
\begin{tabularx}{\textwidth}{|X|c|c|c|}
\hline
\textbf{Módulo} & \textbf{Casos Diseñados} & \textbf{Cobertura Estimada} & \textbf{Estado} \\ \hline
Autenticación Tradicional & 12 & 85\% & \estadowarning \\ \hline
Autenticación Biométrica & 18 & 80\% & \estadopending \\ \hline
Gestión de Viajes (CRUD) & 16 & 90\% & \estadowarning \\ \hline
Clima & 8 & 70\% & \estadopending \\ \hline
Administración & 6 & 75\% & \estadopending \\ \hline
Seguridad General & 10 & Variable & \estadopending \\ \hline
Microservicio Facial & 9 & 80\% & \estadopending \\ \hline
Middlewares & 8 & 85\% & \estadowarning \\ \hline
\textbf{TOTAL} & \textbf{87} & \textbf{82\% promedio} & \textbf{Mixto} \\ \hline
\end{tabularx}
\end{table}

% ============================================================
% 11. PLAN DE ACCIÓN PARA SPRINT DE CALIDAD 3
% ============================================================
\section{Plan de Acción para Sprint de Calidad 3}

\subsection{Objetivos del Sprint 3}

\begin{table}[H]
\centering
\caption{Objetivos y Métricas Objetivo}
\label{tab:objetivos_sprint3}
\begin{tabularx}{\textwidth}{|l|c|c|c|}
\hline
\textbf{Métrica} & \textbf{Actual} & \textbf{Objetivo} & \textbf{Mejora} \\ \hline
Cobertura de Código & 15.3\% & >=70\% & +54.7\% \\ \hline
Issues de Confiabilidad & 94 & <=20 & -74 \\ \hline
Issues de Seguridad & 1 & 0 & -1 \\ \hline
Tasa de Pase de Tests & 34.5\% & >=90\% & +55.5\% \\ \hline
Pruebas E2E Completadas & 9/16 & 16/16 & +7 \\ \hline
Pruebas Integración & 0/28 & 28/28 & +28 \\ \hline
Pruebas Seguridad & 0/10 & 10/10 & +10 \\ \hline
\end{tabularx}
\end{table}

\subsection{Cronograma de Actividades}

\begin{table}[H]
\centering
\caption{Cronograma Sprint de Calidad 3}
\label{tab:cronograma_sprint3}
\begin{tabularx}{\textwidth}{|l|p{8cm}|c|c|}
\hline
\textbf{Semana} & \textbf{Actividades Principales} & \textbf{Horas} & \textbf{Responsable} \\ \hline
Semana 1 & 
1. Corrección de pruebas unitarias backend\newline
2. Mejorar cobertura de services a >70\%\newline
3. Resolver 20 issues de confiabilidad & 20 & Cáceres \\ \hline
Semana 2 & 
1. Ejecutar pruebas de integración (28 casos)\newline
2. Ejecutar pruebas de seguridad OWASP ZAP\newline
3. Implementar pruebas de rendimiento & 18 & Anthony \\ \hline
Semana 3 & 
1. Completar pruebas E2E faltantes (7 casos)\newline
2. Generar reporte final de calidad\newline
3. Validación final con SonarQube & 15 & Ambos \\ \hline
\textbf{TOTAL} & & \textbf{53 horas} & \\ \hline
\end{tabularx}
\end{table}

\subsection{Recursos Necesarios}

\begin{itemize}
    \item \textbf{Herramientas:}
    \begin{itemize}
        \item OWASP ZAP para pruebas de seguridad
        \item Postman para pruebas de integración
        \item Cypress para pruebas E2E adicionales
        \item Jest para mejorar pruebas unitarias
    \end{itemize}
    
    \item \textbf{Entornos:}
    \begin{itemize}
        \item Entorno de desarrollo para correcciones
        \item Entorno de staging para pruebas integración
        \item Base de datos de prueba para pruebas unitarias
    \end{itemize}
    
    \item \textbf{Datos de Prueba:}
    \begin{itemize}
        \item Usuarios de prueba con diferentes roles
        \item Datos de viajes, destinos, itinerarios
        \item Casos de prueba para autenticación biométrica
    \end{itemize}
\end{itemize}

% ============================================================
% 12. CONCLUSIONES Y RECOMENDACIONES
% ============================================================
\section{Conclusiones y Recomendaciones}

\subsection{Logros Alcanzados}

\begin{enumerate}
    \item \estadook \textbf{Infraestructura de Pruebas:} Configurada completamente
    \item \estadook \textbf{Automatización E2E:} 9 flujos críticos automatizados y funcionando
    \item \estadook \textbf{Pruebas Frontend:} Componentes y servicios validados exitosamente
    \item \estadook \textbf{Integración Continua:} SonarQube configurado y monitoreando calidad
    \item \estadook \textbf{Documentación:} Diseño completo de 87 casos de prueba
\end{enumerate}

\subsection{Áreas de Mejora Críticas}

\begin{enumerate}
    \item \estadofailed \textbf{Cobertura de Código:} 15.3\% es insuficiente para producción
    \item \estadofailed \textbf{Pruebas Backend:} 100\% de suites fallando requiere atención inmediata
    \item \estadowarning \textbf{Issues de Confiabilidad:} 94 issues abiertos es riesgo alto
    \item \estadopending \textbf{Pruebas de Seguridad:} No ejecutadas aún (riesgo desconocido)
    \item \estadowarning \textbf{Duplicación de Código:} 2.6\% puede reducir mantenibilidad
\end{enumerate}

\subsection{Recomendaciones Técnicas}

\subsubsection{Inmediatas (Sprint 3):}
\begin{enumerate}
    \item Priorizar corrección de pruebas unitarias backend
    \item Implementar pipeline de CI/CD que bloquee con baja cobertura
    \item Establecer estándar mínimo de 70\% de cobertura
    \item Ejecutar pruebas de seguridad antes de próximo despliegue
\end{enumerate}

\subsubsection{Mediano Plazo:}
\begin{enumerate}
    \item Implementar pruebas de rendimiento y carga
    \item Establecer métricas de calidad como KPI del equipo
    \item Automatizar más flujos E2E críticos
    \item Implementar mutation testing para validar calidad de pruebas
\end{enumerate}

\subsubsection{Largo Plazo:}
\begin{enumerate}
    \item Establecer cultura de "quality first" en el equipo
    \item Implementar trunk-based development con calidad
    \item Automatizar completamente el proceso de release
    \item Establecer programa de refactoring continuo
\end{enumerate}

\subsection{Evaluación General}

\begin{table}[H]
\centering
\caption{Evaluación General del Estado de Calidad}
\label{tab:evaluacion_general}
\begin{tabularx}{\textwidth}{|X|c|c|}
\hline
\textbf{Aspecto} & \textbf{Calificación} & \textbf{Comentarios} \\ \hline
Diseño de Pruebas & 8/10 & 87 casos bien documentados \\ \hline
Ejecución de Pruebas & 4/10 & Muchas pruebas pendientes o fallidas \\ \hline
Cobertura de Código & 2/10 & Muy por debajo del estándar \\ \hline
Automatización & 7/10 & Buena infraestructura, pero incompleta \\ \hline
Seguridad & 5/10 & 1 issue detectado, pruebas pendientes \\ \hline
Documentación & 9/10 & Excelente documentación de pruebas \\ \hline
\textbf{PROMEDIO} & \textbf{5.8/10} & \textbf{Requiere mejora significativa} \\ \hline
\end{tabularx}
\end{table}

\subsection{Conclusión Final}

El proyecto TravelBrain cuenta con una \textbf{excelente base de diseño de pruebas} (87 casos documentados) y \textbf{infraestructura de automatización} configurada. Sin embargo, la \textbf{ejecución real de pruebas} ha revelado problemas críticos que deben atenderse urgentemente:

1. \textbf{Las pruebas unitarias del backend están completamente fallando} - Esto impide validar la lógica central del sistema.

2. \textbf{La cobertura del 15.3\% es inaceptable} para un sistema que maneja datos sensibles de usuarios.

3. \textbf{Los 94 issues de confiabilidad} representan un riesgo alto para la estabilidad en producción.

\textbf{Recomendación:} Concentrar todos los esfuerzos del Sprint 3 en corregir las pruebas unitarias y mejorar la cobertura antes de proceder con nuevas funcionalidades. La calidad actual del código no es suficiente para un despliegue en producción.

% ============================================================
% REFERENCIAS
% ============================================================
\newpage
\printbibliography[title={Referencias Bibliográficas}]

\begin{filecontents}{referencias.bib}
@techreport{ieee829,
author = {{IEEE}},
title = {IEEE Standard for Software and System Test Documentation},
institution = {Institute of Electrical and Electronics Engineers},
year = {2008},
number = {IEEE Std 829-2008}
}

@techreport{iso29119,
author = {{ISO/IEC/IEEE}},
title = {Software and systems engineering -- Software testing -- Part 3: Test documentation},
institution = {International Organization for Standardization},
year = {2013},
number = {ISO/IEC/IEEE 29119-3:2013}
}

@book{jorgensen2013,
author = {Jorgensen, Paul C.},
title = {Software Testing: A Craftsman's Approach},
edition = {4th},
publisher = {CRC Press},
year = {2013},
isbn = {978-1466560680}
}

@misc{owasp2021,
author = {{OWASP Foundation}},
title = {OWASP Top 10 - 2021},
year = {2021},
url = {https://owasp.org/Top10/},
note = {Accessed: 2026-01-21}
}

@misc{sonarqube2026,
author = {{SonarSource}},
title = {SonarQube Documentation},
year = {2026},
url = {https://docs.sonarqube.org/},
note = {Accessed: 2026-01-21}
}

@misc{cypress2026,
author = {{Cypress.io}},
title = {Cypress Documentation},
year = {2026},
url = {https://docs.cypress.io/},
note = {Accessed: 2026-01-21}
}
\end{filecontents}

% ============================================================
% ANEXOS
% ============================================================
\newpage
\appendix
\section{Anexos}

\subsection{Anexo A: Configuración de Cypress}

\begin{lstlisting}[language=JavaScript, caption={cypress.config.js - Configuración Real}]
const { defineConfig } = require("cypress");

module.exports = defineConfig({
  e2e: {
    baseUrl: "https://travelbrain.ddns.net",
    specPattern: "cypress/e2e/**/*.cy.{js,jsx,ts,tsx}",
    supportFile: "cypress/support/e2e.js",
    video: true,
    screenshotOnRunFailure: true,
    viewportWidth: 1280,
    viewportHeight: 720,
    defaultCommandTimeout: 10000,
    requestTimeout: 10000,
    responseTimeout: 10000,
    env: {
      apiUrl: "https://travelbrain.ddns.net/api"
    }
  }
});
\end{lstlisting}

\subsection{Anexo B: Comandos Personalizados Cypress Ejecutados}

\begin{lstlisting}[language=JavaScript, caption={cypress/support/commands.js - Comandos Reales}]
// Login command usado en pruebas
Cypress.Commands.add("login", (email, password) => {
  cy.session([email, password], () => {
    cy.visit('/login')
    cy.get('input[name="email"]').type(email)
    cy.get('input[type="password"]').type(password)
    cy.get('button[type="submit"]').click()
    cy.url().should('include', '/dashboard')
  })
})

// Comando para crear viaje
Cypress.Commands.add("createTrip", (tripData) => {
  cy.visit('/trips')
  cy.contains('New Trip').click()
  
  cy.get('input[name="title"]').type(tripData.title)
  cy.get('input[name="destination"]').type(tripData.destination)
  cy.get('input[name="startDate"]').type(tripData.startDate)
  cy.get('input[name="endDate"]').type(tripData.endDate)
  cy.get('input[name="budget"]').type(tripData.budget)
  
  cy.get('button[type="submit"]').click()
  cy.contains('Trip created successfully').should('be.visible')
})
\end{lstlisting}

\subsection{Anexo C: Configuración de Jest para Backend}

\begin{lstlisting}[language=JavaScript, caption={jest.config.js - Backend Project}]
module.exports = {
  testEnvironment: 'node',
  roots: ['<rootDir>/tests'],
  testMatch: ['**/*.test.js'],
  collectCoverage: true,
  coverageDirectory: 'coverage',
  coverageReporters: ['text', 'lcov', 'html'],
  collectCoverageFrom: [
    'src/**/*.js',
    '!src/**/*.test.js',
    '!src/**/index.js'
  ],
  coverageThreshold: {
    global: {
      branches: 70,
      functions: 70,
      lines: 70,
      statements: 70
    }
  },
  setupFilesAfterEnv: ['<rootDir>/tests/setup.js']
};
\end{lstlisting}

\subsection{Anexo D: Reporte de SonarQube (Extracto)}

\begin{lstlisting}[caption={Resultados SonarQube - Extracto}]
QUALITY GATE: PASSED (with warnings)

=== METRICS ===
Lines of Code: 19,001
Coverage: 15.3%
Duplications: 2.6%
Issues: 381
  - Security: 1
  - Reliability: 94 (C - Critical)
  - Maintainability: 286 (A - Acceptable)

=== MODULE ANALYSIS ===
backend-project/src:
  Lines: 3,064
  Coverage: 17.2%
  Issues: 31

business-rules-backend/src:
  Lines: 1,505
  Coverage: 7.5%
  Issues: 108

frontend-react/src:
  Lines: 14,432
  Coverage: 17.1%
  Issues: 242

=== RECOMMENDATIONS ===
1. Increase code coverage to at least 70%
2. Fix critical reliability issues
3. Reduce code duplication
4. Implement security testing
\end{lstlisting}

\subsection{Anexo E: Pruebas E2E Ejecutadas - Evidencia}

\begin{lstlisting}[caption={Resumen Pruebas E2E Ejecutadas}]
Tests Ejecutados: 9
Passed: 9
Failed: 0
Duration: ~3 minutos

Detalle de pruebas:
1. [OK] travelbrain.e2e.cy.js
   - Carga la página principal
   - Login de usuario exitoso
   - Login fallido con credenciales incorrectas

2. [OK] trips.e2e.cy.js
   - Crear un viaje exitosamente
   - Navegar al dashboard después del login

3. [OK] navigation.e2e.cy.js
   - Acceder a la página de destinos
   - Acceder a la página del clima
   - Acceder a la página de perfil
   - Verificar protección de rutas sin autenticación

URLs probadas:
- https://travelbrain.ddns.net/
- https://travelbrain.ddns.net/login
- https://travelbrain.ddns.net/dashboard
- https://travelbrain.ddns.net/trips
- https://travelbrain.ddns.net/destinations
- https://travelbrain.ddns.net/weather
- https://travelbrain.ddns.net/profile

Credenciales usadas:
- Email: test@test.com
- Password: admin1234
\end{lstlisting}

\subsection{Anexo F: Issues de SonarQube - Ejemplos}

\begin{table}[H]
\centering
\caption{Ejemplos de Issues Detectados por SonarQube}
\label{tab:issues_ejemplos}
\begin{tabularx}{\textwidth}{|l|p{10cm}|c|}
\hline
\textbf{Tipo} & \textbf{Descripción} & \textbf{Severidad} \\ \hline
Vulnerabilidad & Hardcoded credentials in configuration file & MEDIUM \\ \hline
Bug & Possible null pointer dereference in userController & MAJOR \\ \hline
Code Smell & Function has too many parameters (8) & MINOR \\ \hline
Duplicación & Same logic found in tripService and itineraryService & MAJOR \\ \hline
Cobertura & Method calculateBudget() has 0\% coverage & CRITICAL \\ \hline
Seguridad & Missing input validation in register endpoint & MAJOR \\ \hline
\end{tabularx}
\end{table}

\end{document}